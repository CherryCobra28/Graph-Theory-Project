\documentclass{article}
\usepackage{amsmath}
\usepackage[style=authoryear-ibid,backend=biber]{biblatex}
\title{SIRD Model over Small World, Scale Free Graphs}
\date{12/11/2022}
\author{Thomas Draycott}
\addbibresource{references.bib}
\begin{document}
    \maketitle
    \pagenumbering{arabic}
    \newpage
    \section{Introduction}
    TOBEADDED
    


    \section{Literature Review}
        \subsection{Graph Theory}
        Graph Theory is the study of networks (graphs will be referred to as networks from now on) where vertices (we will also from this point forwards we will be referring to vertices as nodes) are connected by edges, see APPENDIX for a further explanation.
        \subsection{Small World Graphs}
        In May 1967 Professor of Psychology at the Graduate School and University Center of the City University of New York, Stanley Milgram ran an experiment to see if a person living
        in Omaha, Nebraska could get a parcel to a stockbroker in Boston, Massachusetts \parencite{milgram1967small}. In his experiment he found the average path length to reach the stockbroker was 5.5, which created the term
        six degrees of separation (however Milgram's experiment had flaws which puts the exact number into doubt). This idea of having such a small average path length for such numerous nodes is a hallmark of a small world graph.
        \\A Small world graph is defined by the following property: $L\propto\log{N}$ where $L$ is the average path length of the network and $N$ is the total number of nodes \parencite{Watts1998}.

    
    
    
    
\printbibliography
\end{document}